\documentclass[11pt, letterpaper]{article}
\title{Homework5}
\author{Brian Bowers}
\date{\today}

\usepackage{enumitem}
\usepackage{listings}
\usepackage[margin=1.0in]{geometry}
\usepackage{tikz}
\usepackage{tikz}
\usetikzlibrary{automata, positioning}
\usepackage{hyperref}
\usepackage{tcolorbox}

\usepackage{listings}
\lstset{
    basicstyle=\ttfamily\small,
    frame=single,
    columns=fullflexible
}


\begin{document}

\begin{flushright}
    Homework 5 \\
    May 8, 2025
\end{flushright}

\noindent{Group: August Wetterau, Brian Bowers, Thomas Rife, Walter Campos}\\

\noindent{Project: \url{https://github.com/thomas-rife/PANIC}} \\

\noindent{\underline{AW}} I read through all the assigned readings, worked through the course notes, read the book chapters, and browsed the provided articles.

\noindent{\underline{BB}} I read through all the assigned readings, worked through the course notes, read the book chapters, and browsed the provided articles.

\noindent{\underline{TR}} I read through all the assigned readings, worked through the course notes, read the book chapters, and browsed the provided articles.

\noindent{\underline{WC}} I read through all the assigned readings, worked through the course notes, read the book chapters, and browsed the provided articles. \\

\begin{enumerate}
    \item The answers to problem 1 can be found here: \url{https://github.com/Bbowers2/3802_homework_5/blob/main/src/regex_exercises.js}.
    
    \item WebAssembly:
    \begin{lstlisting}
        local.get       0       ;; get input number noindent
        i32.const       3       ;; push 3
        i32.mul                 ;; n*3
        i32.const       1       ;; push 1
        i32.add                 ;; add it (3n + 1)
        local.get       0       ;; get n again
        i32.const       1       ;; push 1
        i32.shr_s               ;; shift right by 1 (same as n/2)
        local.get       0       ;; get n again
        i32.const       1       ;; push 1
        i32.and                 ;; n & 1
        i32.select              ;; if odd use 3*n + 1 else use n/2
        end_function
        \end{lstlisting}

    x86-64:
    \begin{lstlisting}
        mov     ecx, edi                ;; copy lower 32 bits (int) of rdi into ecx
        sar     ecx                     ;; shift right by one, which is n/2
        test    dil, 1                  ;; check if last bit is even (0) or odd (1)
        lea     eax, [rdi + 2*rdi + 1]  ;; puts 3*n + 1 into eax
        cmove   eax, ecx                ;; if n is even move n/2 (ecx) into eax otherwise keep 3*n+1
        ret                             ;; return eax
        \end{lstlisting}

    \item Starting with any $\langle M, w \rangle$ from the undecidable acceptance problem, two machines can be constructed, $\langle M_1, M_2 \rangle$, where $M_1$ runs $M$ on input $w$, then accepts its own input if and only if that simulation accepts. Machine $M_2$ rejects every input unconditionally. If $M$ accepts $w$, then $L(M_1) = \Sigma^*$ and $L(M_2) = \emptyset$. If $M$ does not accept $w$, then both $M_1$ and $M_2$ accept no strings, so $L(M_1) = L(M_2) = \emptyset$. Thus, $L(M_1) = L(M_2)$ if and only if $M$ does not accept $w$. Therefore, if we had a decider for testing language equivalence, we could use it to decide whether $M$ accepts $w$. Since the acceptance problem is undecidable, the language equivalence problem must also be undecidable.
\end{enumerate}

\end{document};